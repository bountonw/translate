% =============================================================================
% DOCUMENT CLASS AND GEOMETRY
% =============================================================================
% Book class with mirrored margins, no forced right-hand openings
\documentclass[10pt,twoside,openright]{memoir}

\setSingleSpace{1.23}

\setstocksize{210mm}{148mm}
\settrimmedsize{210mm}{148mm}{*}
\settrims{0pt}{0pt}

\setlrmarginsandblock{21mm}{12mm}{*}
\setulmarginsandblock{19mm}{23mm}{*}

\setheadfoot{14pt}{10mm}
\setheaderspaces{*}{12pt}{*}

\checkandfixthelayout

\usepackage{showframe}

\raggedbottom

% =============================================================================
% BASELINE AND SPACING FUNDAMENTALS
% =============================================================================
% Rigid 12.6pt baseline - remove all stretch/shrink
% \setlength{\baselineskip}{12.6pt}
% \setlength{\parskip}{0pt}
% \setlength{\parindent}{1.5em}
% \setlength{\lineskip}{0pt}
% \setlength{\lineskiplimit}{-\maxdimen}
% \setlength{\abovedisplayskip}{12.6pt}
% \setlength{\belowdisplayskip}{12.6pt}
% \setlength{\abovedisplayshortskip}{12.6pt}
% \setlength{\belowdisplayshortskip}{12.6pt}
% \setlength{\topskip}{12.6pt}


% =============================================================================
% FONT SETUP
% =============================================================================
\usepackage{fontspec}
\usepackage{microtype}

% -----------------------------------------------------------------------------
% Lao Font: Saysettha (Primary Lao Font)
% -----------------------------------------------------------------------------
\newfontfamily\laofont{Saysettha}[
    Path = ../../assets/fonts/saysettha/,
    Extension = .ttf,
    UprightFont = *-Regular,
    ItalicFont = *-Italic,
    BoldFont = *-Bold,
    BoldItalicFont = *-BoldItalic
]

% Setup separate fallbacks for each font style
\directlua{
  luaotfload.add_fallback(
    "laofallback",
    {"[../../assets/fonts/saysettha/Saysettha-Regular.ttf]:mode=harf;"}
  )
  luaotfload.add_fallback(
    "laofallbackbold",
    {"[../../assets/fonts/saysettha/Saysettha-Bold.ttf]:mode=harf;"}
  )
  luaotfload.add_fallback(
    "laofallbackitalic",
    {"[../../assets/fonts/saysettha/Saysettha-Italic.ttf]:mode=harf;"}
  )
  luaotfload.add_fallback(
    "laofallbackbolditalic",
    {"[../../assets/fonts/saysettha/Saysettha-BoldItalic.ttf]:mode=harf;"}
  )
}

% -----------------------------------------------------------------------------
% Main Font: Libertinus Serif with Lao Fallback
% -----------------------------------------------------------------------------
\setmainfont{Libertinus Serif}[
  RawFeature={fallback=laofallback},
  BoldFeatures={RawFeature={fallback=laofallbackbold}},
  ItalicFeatures={RawFeature={fallback=laofallbackitalic}},
  BoldItalicFeatures={RawFeature={fallback=laofallbackbolditalic}}
]

% -----------------------------------------------------------------------------
% Thai Font: Sarabun
% -----------------------------------------------------------------------------
\newfontfamily\thaifont{Sarabun}[
    Path = ../../../th/assets/fonts/saraban/,
    Extension = .ttf,
    UprightFont = *-Regular,
    ItalicFont = *-Italic,
    BoldFont = *-Bold,
    BoldItalicFont = *-BoldItalic,
    Script = Thai
]

\newcommand{\thai}[1]{{\thaifont #1}}

% -----------------------------------------------------------------------------
% Font Size and Microtypography
% -----------------------------------------------------------------------------
% Set document font size (applied at document start)
\AtBeginDocument{\fontsize{10.5pt}{12.6pt}\selectfont}

% Customize superscript positioning
\renewcommand{\textsuperscript}[1]{%
  {\fontsize{7}{7}\selectfont\raisebox{1.4ex}{#1}}%
}

% Enable font expansion; disable protrusion to avoid TU-slot warnings
\microtypesetup{expansion=true,protrusion=false}

% Font expansion configuration
\SetExpansion[stretch=10,shrink=10,step=1,auto=true]{encoding=TU,family=*}{}


% =============================================================================
% HEADERS AND FOOTERS
% =============================================================================
\usepackage{fancyhdr}
\usepackage{url}

\newcommand{\booktitle}{ປາກທາງແຫ່ງຄວາມຫວັງ}
\newcommand{\currentchaptertitle}{}  % Updated by \laochapter

\setlength{\headheight}{14pt}
\pagestyle{fancy}
\fancyhf{}

% Header configuration (using existing FakeSlant from \emph)
\fancyhead[LE]{\thepage}
\fancyhead[CE]{\emph{\booktitle}}
\fancyhead[CO]{\emph{\currentchaptertitle}}
\fancyhead[RO]{\thepage}

\renewcommand{\headrulewidth}{0pt}

% Chapter-opening page style (for source URL footer)
\fancypagestyle{chapterstart}{%
  \fancyhf{}%
  \fancyfoot[C]{%
    \begingroup
    \footnotesize\itshape
    ຕົ້ນສະບັບພາສາອັງກິດ:\ %
    \def\UrlFont{\itshape\rmfamily}%
    \expandafter\url\expandafter{\the\ThisChapterSourceToks}%
    \endgroup
  }%
  \renewcommand{\headrulewidth}{0pt}%
  \renewcommand{\footrulewidth}{0pt}%
} 


% =============================================================================
% TOC
% =============================================================================

\settocdepth{chapter}
\renewcommand{\contentsname}{ສາລະບານ}

\renewcommand{\cftchapterfont}{\normalfont}
\renewcommand{\cftchapterpagefont}{\normalfont}

\setlength{\cftbeforechapterskip}{0pt}
\cftsetindents{chapter}{12.5em}{2.5em}

\setpnumwidth{12mm}
\setrmarg{13mm}

\renewcommand{\cftchapterdotsep}{1.5}
\renewcommand{\cftchapterfillnum}{\cftdotfill{\cftchapterdotsep}}
\renewcommand{\cftchapteraftersnum}{.\hspace{1.5em}}

\makeatletter
\renewcommand{\cftchapterformatpnum}[1]{\hb@xt@\@pnumwidth{\hfil#1}}
\makeatother

% Wrapper for local TOC settings
\makeatletter
\let\origtableofcontents\tableofcontents
\renewcommand{\tableofcontents}{%
  \begingroup
    \setrmarg{12mm}
    \setpnumwidth{12mm}
    \origtableofcontents
  \endgroup
}
\makeatother


% % =============================================================================
% % TOC FORMATTING
% % =============================================================================
% \usepackage{tocloft}
% 
% % Chapters only in TOC (no sections or subsections)
% \setcounter{tocdepth}{0}
% 
% % Remove bold from page numbers
% \renewcommand{\cftchappagefont}{\normalfont}
% 
% % Add period after chapter number
% \renewcommand{\cftchapaftersnum}{.}
% 
% % Set chapter number width
% \setlength{\cftchapnumwidth}{2.5em}
% 
% % Add leader dots between title and page number
% \renewcommand{\cftchapleader}{\cftdotfill{\cftdotsep}}
% 
% % Reduce line spacing between TOC entries
% \setlength{\cftbeforechapskip}{0pt}
% 
% % Right-align page numbers 0.5 inch (12.7mm) from right text edge
% \makeatletter
% \renewcommand{\cftchappagefont}{%
%   \hspace*{0pt plus 1fil}%
%   \makebox[12.7mm][r]{\normalfont}%
% }
% \makeatother
% 
% % Customize TOC title using chapter formatting
% \renewcommand{\contentsname}{ສາລະບານ}

% =============================================================================
% CHAPTER FORMATTING
% =============================================================================
\usepackage{titlesec}

% Chapter spacing: minimal gap between number and title
\titlespacing*{\chapter}{0pt}{3.38pt}{0.5\baselineskip}

% Chapter format: centered, with large number and bold title
\titleformat{\chapter}[display]
{\normalfont\centering}
{{\ChapNumberSize \thechapter}}
{-0.7\baselineskip}
{\ChapTitleSize}

% Size controls for chapter elements
\newcommand{\ChapNumberSize}{\fontsize{60pt}{66pt}\selectfont}
\providecommand{\ChapTitleSize}{}
\renewcommand{\ChapTitleSize}{\fontsize{15.5pt}{15.5pt}\selectfont\bfseries}

% Utility to strip \lw{} wrappers in chapter titles for headers
\makeatletter
\providecommand{\lw@title}[1]{#1}
\AtBeginDocument{%
  \@ifundefined{lw}{}{%
    \let\LW@orig\lw
    \renewcommand{\lw@title}[1]{\begingroup\LW@orig{#1}\unskip\endgroup}%
  }%
}
\makeatother


% =============================================================================
% SECTION FORMATTING
% =============================================================================
% Suppress section numbering (keep chapters numbered)
\setcounter{secnumdepth}{0}

% Grid-aligned sections: exactly 2 baselines (25.2pt total)
% Font: 12pt, Spacing: 9pt before + 12pt font + 4.2pt after = 25.2pt
\makeatletter
\@ifundefined{KOMAClassName}{%
  % Standard classes
  \titleformat{\section}
    {\normalfont\fontsize{12pt}{12pt}\selectfont\bfseries}
    {\thesection}{1em}{}
  \titlespacing*{\section}{0pt}{9pt}{4.2pt}
  
  \renewcommand\subsection{%
    \@startsection{subsection}{2}{\z@}%
      {1.25ex plus .4ex minus .2ex}%
      {.6ex plus .2ex}%
      {\normalfont\normalsize\bfseries}%
  }%
}{%
  % KOMA-Script
  \RedeclareSectionCommand[
    beforeskip=9pt,
    afterskip=4.2pt,
    font=\normalfont\fontsize{12pt}{12pt}\selectfont\bfseries
  ]{section}
  \RedeclareSectionCommand[
    beforeskip=1.25ex plus .4ex minus .2ex,
    afterskip=.6ex plus .2ex,
    font=\normalfont\normalsize\bfseries
  ]{subsection}
}%
\makeatother


% =============================================================================
% PARAGRAPH INDENTATION
% =============================================================================
% Indent first paragraph after headings (global setting)
\makeatletter
\let\@afterindentfalse\@afterindenttrue
\@afterindenttrue
\makeatother


% =============================================================================
% EMPHASIS SEQUENCE
% =============================================================================
% Define emphasis nesting: italic → upright → italic → upright
\DeclareEmphSequence{%
  {\itshape},%
  {\upshape},%
  {\itshape},%
  {\upshape}%
}


% =============================================================================
% URL STYLING
% =============================================================================
\urlstyle{same}  % URLs use same font as surrounding text
