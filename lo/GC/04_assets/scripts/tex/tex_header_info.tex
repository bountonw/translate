% Strong discouragement of widows/orphans
\widowpenalty=8000
\clubpenalty=8000


\makeatletter
% Basic footnote mark: superscripted number in text
\renewcommand{\@makefnmark}{\hbox{\textsuperscript{\normalfont\@thefnmark}}}

% Basic footnote text: number as a superscript, then a space, then the text
\renewcommand{\@makefntext}[1]{%
  \parindent 1em%
  \noindent
  \hb@xt@1.8em{\hss\textsuperscript{\normalfont\@thefnmark}}#1%
}
\makeatother

% Flexible spaces (primary solution)
\spaceskip=.6em plus .4em minus .2em
\xspaceskip=\spaceskip

% Moderate tolerance (don't need extreme values now)
\tolerance=2400

% Minimal emergency stretch (probably won't be needed)
\emergencystretch=2em

% \textspace — insert a Lao “phrasal” space that is elastic and safe around \nobreak
% Removes any prior interword glue/penalty, then inserts moderately elastic glue.
% The trailing 0pt skip is a sacrificial sentinel so a following \nobreak will
% unskip *that* instead of erasing the real visible space.
\protected\def\textspace{%
  \ifhmode \unskip \unpenalty \fi
  \penalty-80 %
  \hskip .60em plus 0.50em minus .10em %
  \hskip 0pt\relax
}

% \nobreak — hard no-break barrier for Lao text
% If used after glue, remove that glue first, then set an infinite penalty.
\protected\def\nobreak{%
  \ifhmode\unskip\fi
  \penalty10000\relax
}


% Penalty system for line breaking
\makeatletter
\@ifundefined{p}{%
  \newcommand{\p}[1]{\penalty #1\relax}%
}{%
  \renewcommand{\p}[1]{\penalty #1\relax}%
}
\makeatother


% Font Expansion
\microtypesetup{expansion=true}
\SetExpansion[stretch=10,shrink=10,step=1,auto=true]{encoding=TU,family=*}{}

%%%%%%%%%%%%%%%%%%
% Lao word wrapper
\newcount\lwpenalty
\lwpenalty=-300
\newcommand{\lw}[1]{#1\penalty\lwpenalty\hskip 0pt plus .12em minus .04em\relax}


% Disable TeX hyphenation fallbacks
\hyphenpenalty=10000
\exhyphenpenalty=10000
\uchyph=0


% Spacing commands
\newcommand{\zwsp}{\hspace{0pt}\allowbreak}  % Zero-width space with break opportunity

\protected\def\nbsp{%
  \ifhmode\unskip\fi
  \nobreak\kern .30em\relax
}

\protected\def\cs{\hspace{0.4em plus 0.2em minus 0.1em}\allowbreak}  % Compound space
% Flexible and rigid spacing for Lao typography
% Source: \s marker in .md files converts to \fs (More flexible spacing)
\newcommand{\flexspace}{\penalty-300\hspace{0pt plus 3em}\relax}
% Source: \S marker in .md files converts to \rs (Less flexible spacing)
\newcommand{\rigidspace}{\penalty-300\hspace{0pt plus 1.5em}\relax} 

\newcommand{\scripturespace}{\p{200}\hskip 0.8em plus 0.4em minus 0.1em\relax}
\newcommand{\scripturereference}[1]{\mbox{#1}}

% Short aliases for typing efficiency  
\let\fs\flexspace    % \s → \fs{} (flex space)
\let\rs\rigidspace   % \S → \rs{} (rigid space)
\let\scrspace\scripturespace
\let\scrref\scripturereference

%%%%%%%%%%%%%%%%%%%%%%%%%%%%%%%%%%%%%%%%%%%%%%%%%%%%%%%%%%%%%%%%%%%%%%%%%%%%%%
% === Lao book tweaks: numbering, chapter title handling, and sane spacing ===


% === Tighten vertical spacing around chapter and first section ===
% \titlespacing{<command>}{left}{before-sep}{after-sep}
\titlespacing*{\chapter}{0pt}{0pt}{0.50\baselineskip}
\titlespacing*{\section}{0pt}{0.5\baselineskip}{0.25\baselineskip}

% Indent first paragraph after headings (global)
\makeatletter
\let\@afterindentfalse\@afterindenttrue
\@afterindenttrue
\makeatother

% Lao Chapter macro: keep first-par indent and suppress folio on opening page

\makeatletter
\newcommand{\laochapter}[2]{%
  \setcounter{chapter}{\numexpr #1-1\relax}%
  % Store cleaned version for headers only
  \begingroup
    \def\lw##1{##1}% Strip \lw{} wrappers
    \def\p##1{}% Strip \p{} penalty commands
    \xdef\currentchaptertitle{#2}% Store globally cleaned version
  \endgroup
  \par %
  \begingroup
    \let\orig@lw\lw
    \def\lw##1{##1\ }% plain content + a space for the heading only
    \@afterindenttrue
    \chapter{#2}% Original #2 with all \lw{} and \p{} codes intact
    \thispagestyle{empty}% no page number on the chapter-opening page
  \endgroup
}
\makeatother

% If you also want section *titles* un-numbered in the TOC, add:
% \setcounter{tocdepth}{0}  % (optional) only chapters in the TOC

% Chapter and reference commands
% \newcommand{\laochapter}[2]{%
%   \clearpage%
%   \thispagestyle{empty}%
%   \vspace*{2cm}%
%   \begin{center}%
%     {\Huge\textbf{#1}}\\[2em]%
%     {\LARGE\textbf{#2}}%
%   \end{center}%
%   \vspace{3cm}%
%   \setcounter{chapter}{#1}%
% }


% Initialize folios at document start (plain footer, Arabic, start at 1)
\newcommand{\GCInitFolios}{%
  \pagestyle{fancy}%
  \pagenumbering{arabic}%
  \setcounter{page}{1}%
}

% === Chapter-opening footer for source URL (first page only; silent inline) ===
% \source{<url>} prints nothing in the body and sets a centered footer
% on the chapter’s opening page only.

\makeatletter
\@ifpackageloaded{url}{}{%
  \RequirePackage{url}%
}
\@ifpackageloaded{fancyhdr}{}{%
  \RequirePackage{fancyhdr}%
}
\makeatother

% Compile-time switch (enabled by default)
\newif\ifprintsourcefoot
\printsourcefoottrue

% Storage for the current chapter's URL (safe for #, %, etc.)
\ifx\ThisChapterSourceToks\undefined
  \newtoks\ThisChapterSourceToks
\fi
\global\ThisChapterSourceToks={}

% Footer style used only on the chapter's first page
\fancypagestyle{chapterstart}{%
  \fancyhf{}%
  % Lao label + actual URL (force expansion of toks into \url{...})

\fancyfoot[C]{%
  \begingroup
  \footnotesize
    {\itshape ຕົ້ນສະບັບພາສາອັງກິດ:}\ %
  \expandafter\url\expandafter{\the\ThisChapterSourceToks}%
  \endgroup
}%

  \renewcommand{\headrulewidth}{0pt}%
  \renewcommand{\footrulewidth}{0pt}%
}

% Define or replace \source to be silent inline and set the opener's footer
\makeatletter
\@ifundefined{source}{%
  \newcommand{\source}[1]{%
    \global\ThisChapterSourceToks={#1}%
    \ifprintsourcefoot \thispagestyle{chapterstart}\fi
  }%
}{%
  \renewcommand{\source}[1]{%
    \global\ThisChapterSourceToks={#1}%
    \ifprintsourcefoot \thispagestyle{chapterstart}\fi
  }%
}
\makeatother



%%%%%%%%%%%%%%%%%%%%%%%%%%%%%%%%%%%%%%%%%%%%%%%%%%%%%%
% Sections and Subsections
% Heading size/spacing (KOMA-aware); keeps first-par indent via positive beforeskip
\makeatletter
\@ifundefined{KOMAClassName}{%
  % Standard classes
  \renewcommand\section{%
    \@startsection{section}{1}{\z@}%
      {1.5ex plus .5ex minus .2ex}%
      {.8ex plus .2ex}%
      {\normalfont\large\bfseries}%
  }%
  \renewcommand\subsection{%
    \@startsection{subsection}{2}{\z@}%
      {1.25ex plus .4ex minus .2ex}%
      {.6ex plus .2ex}%
      {\normalfont\normalsize\bfseries}%
  }%
}{%
  % KOMA-Script
  \RedeclareSectionCommand[
    beforeskip=1.5ex plus .5ex minus .2ex,
    afterskip=.8ex plus .2ex,
    font=\normalfont\large\bfseries
  ]{section}
  \RedeclareSectionCommand[
    beforeskip=1.25ex plus .4ex minus .2ex,
    afterskip=.6ex plus .2ex,
    font=\normalfont\normalsize\bfseries
  ]{subsection}
}%
\makeatother

% Suppress forced blank left pages before chapters (KOMA and non-KOMA)
\makeatletter
\@ifundefined{KOMAClassName}{%
  % Standard book/report insert blanks for openright; make double-clear a single clear
  \let\cleardoublepage\clearpage
}{%
  % KOMA: allow chapters to open on any page (no inserted blanks)
  \KOMAoptions{open=any}
}%
\makeatother

% Normalize page style/numbering at document start without touching the body
\makeatletter
\AtBeginDocument{%
  \pagestyle{fancy}%
  \pagenumbering{arabic}% 1,2,3...
  \setcounter{page}{1}% start at 1
}%
\makeatother


%%%%%%%%%%%%%%%%%%%%%%
% \egw{}
\protected\def\egw#1{%
  \ifhmode\unskip\fi
  \nobreak\hskip0pt\kern .50em%
  \mbox{%
    \fontsize{0.8em}{0.8em}\selectfont
    \textcolor[gray]{0.3}{%
      {\itshape\{#1\}}%
    }%
  }%
  \kern .30em\nobreak\hskip0pt\relax
}

\DeclareEmphSequence{%
  {\itshape},%
  {\upshape},%
  {\itshape},%
  {\upshape}%
}


% === Lao Ellipsis Macros (paste into tex_header_info.tex; replace old defs) ===
% Goal: three-dot ellipsis with *nano* stretch between dots, kept nonbreaking,
%       and visually stable (no margin protrusion). Works with LuaLaTeX+fontspec.

% Core: three literal dots with tiny (rubber) spaces between them.
% - Default internal gap = 0.22em
% - Stretch/shrink = +0.02em / -0.01em  (tune here if desired)
% - Wrapped in \mbox to keep the three dots together (no line break inside)
% - microtype protrusion disabled locally so dots don't hang into the margin
\DeclareRobustCommand{\laoellipsisrubber}[1][0.22em]{%
  {\microtypesetup{protrusion=false}%
   \mbox{.\hspace{#1 plus .02em minus .01em}.\hspace{#1 plus .02em minus .01em}.}}%
}

% Public API (mirrors your original commands), all preserved as nonbreaking:
% - \ellipsis   : side spacing + core
% - \ellbefore  : core + following spacing
% - \ellafter   : leading spacing + core
% - \ellall     : just the core
%
% Side spacing uses gentle rubber glue so it can participate in justification.
\DeclareRobustCommand{\ellipsis}{%
  \hspace{0.2em plus 0.1em minus 0.05em}%
  \mbox{\laoellipsisrubber}%
  \hspace{0.2em plus 0.1em minus 0.05em}%
}

\DeclareRobustCommand{\ellbefore}{%
  \mbox{\laoellipsisrubber}%
  \hspace{0.2em plus 0.1em minus 0.05em}%
}

\DeclareRobustCommand{\ellafter}{%
  \hspace{0.2em plus 0.1em minus 0.05em}%
  \mbox{\laoellipsisrubber}%
}

\DeclareRobustCommand{\ellall}{%
  \mbox{\laoellipsisrubber}%
}

% ---------------------------
% Tuning notes:
% - To make the internal dots slightly looser/tighter across the book,
%   change the default in \laoellipsisrubber above (e.g., [0.24em] or [0.20em]).
% - To constrain stretching even more, reduce "plus .02em minus .01em"
%   (e.g., "plus .01em minus .005em").
% - If you ever want the ellipsis to be breakable after it, remove the \mbox
%   in \ellafter (keep it in the core).
% ---------------------------
% Redefine \nobreak to also delete the preceding interword glue (if any).
% Horizontal mode: remove last glue, then forbid a break; vertical: just forbid.
\protected\def\nobreak{%
  \ifhmode \unskip \unpenalty \fi
  \penalty10000\relax
}

% Lao Repeat
\newcommand{\laorepeat}{\nobreak{}ໆ\penalty-400\relax}
\newcommand{\laorepeatbefore}{\nobreak{}ໆ}


% Define \nodict{} command for highlighting missing dictionary words
% Makes text bold, 1.2em size, and red color (#D13D1D)
\usepackage{xcolor}
\definecolor{nodictcolor}{HTML}{D13D1D}
\newcommand{\nodict}[1]{{\fontsize{1.2em}{1.44em}\selectfont\textbf{\textcolor{nodictcolor}{#1}}}}


% Poetry
\newcommand{\attrib}[1]{\hfill{#1}}
\newcommand{\verseindent}[1]{\hspace{2em}#1}
\newcommand{\verseindentii}[1]{\hspace{4em}#1}
\newcommand{\verseindentiii}[1]{\hspace{6em}#1}
\let\rs\rigidspace
