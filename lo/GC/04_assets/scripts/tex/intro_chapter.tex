\laochapter{0}{ຄຳນຳ}

**ຄຳພະຍາກອນທີ່ຈວນຈະສຳເລັດ**

ເນື່ອງຈາກເຫດການທີ່ທຳນວາຍໄວ້ໃນພຣະຄຣິສຕະທຳຄຳພີເລື່ອງຍຸກສຸດທ້າຍ ໄດ້ສຳເລັດໄປແລ້ວຫຼາຍຢ່າງ, ແລະເບິ່ງຄື ໂລກມະນຸດເຮົາກຳລັງກ້າວເຂົ້າສູ່ສາກສຸດທ້າຍໃນຄຳພະຍາກອນ ກ່ອນທີ່ອົງພຣະເຢຊູຄຣິສຈະສະເດັດກັບມາ, ຜູ້ຈັດພິມຈຶ່ງເຫັນສົມຄວນທີ່ຄົນລາວຈະບໍ່ພາດໂອກາດໃນການຮັບຮູ້ເຖິງຄຳພະຍາກອນທີ່ກ່ຽວຂ້ອງ. *ປາຍທາງແຫ່ງຄວາມຫວັງ* ເປັນໜັງສືທີ່ດີທີ່ສຸດເຫຼັ້ມໜຶ່ງທີ່ອະທິບາຍຄຳພຍາກອນ ແລະ ເປີດເຜີຍສາກເບື້ອງຫຼັງເຫດການຕ່າງໆໃນປະຫວັດສາດ ທີ່ຈະຊ່ວຍບັນດາຜູ້ຫິວກະຫາຍຄວາມຈິງຮູ້ທັນກົນອຸບາຍຂອງຊາຕານ ເຜື້ອຈະບໍ່ຖືກຫຼອກລວງ.ຜູ້ຂຽນໄດ້ເລືອກເຫດການສຳຄັນໆຕັ້ງແຕ່ການທຳລາຍນະຄອນເຢຣູຊາເລັມໃນປີ ຄ.ສ. 70, ການທີ່ຄົນຂອງພຣະເຈົ້າຫຼົງຈາກຄວາມເຊື່ອໃນສະຕະວັດຕໍ່ມາ ເຊິ່ງນຳໄປສູ່ການຂຶ້ນສູ່ອຳນາດຂອງ "ສັດຮ້າຍ" ຕາມຄຳພະຍາກອນ ແລະ ສະໄໝຍຸກມືດ; ນອກຈາກນັ້ນ, ໜັງສືເຫຼັ້ມນີ້ຍັງໄດ້ບັນລະຍາຍເຖິງການດິ້ນລົນຕໍ່ສູ້ລະຫວ່າງຄວາມຈິງກັບຄວາມເທັດ ທີ່ສືບຕໍ່ເນື່ອງກັນມາ ຈົນເຖິງສະໄໝຂອງພວກເຮົາ ທີ່ຂ່າວທູດສະຫວັນສາມອົງກຳລັງໄດ້ຮັບການປ່າວປະກາດ ກ່ອນທີ່ໂອກາດແຫ່ງພຣະກະລຸນາຈະໝົດໄປ.

**ພຣະຄຳພີທີ່ນຳໃຊ້ໃນການອ້າງອີງ**

ພຣະຄຣິສຕະທຳຄຳພີແບ່ງອອກເປັນສ່ອງໝວດໃຫຍ່ ທີ່ເຮັາມັກເອີ້ນກັນວ່າ “ພຣະຄຳພີເດີມ” ແລະ “ພຣະຄຳພີໃໝ່.” ໃນສະໄໝດັ້ງເດີມ, ພຣະຄຳພີເດີມຂຽນເປັນພາສາເຮັບເຣີ; ສ່ວນພຣະຄຳພີໃໝ່ ກໍຂຽນເປັນພາສາກຣີກ. ຕໍ່ມາມີການແປພຣະຄຳພີເປັນພາສາຕ່າງໆ ແລະ ໃນຫຼາຍພາສາມີການຈັດແປຫຼາຍຄັ້ງຫຼາຍລຸ້ນ ທີ່ເຮົາເອີ້ນວ່າ “version” ຫຼືວ່າ “ສະບັບແປ.” ຜູ້ຂຽນສ່ວນຫຼາຍໃຊ້ສະບັບ ຄິງເຈມ (KJV) ເຊິ່ງເປັນສະບັບແປພາສາອັງກິດທີ່ເກົ່າແກ່; ແລະເພື່ອຮັກສາແນວຄິດ ແລະ ຈິນຕະນາລົມຂອງຜູ້ຂຽນໃຫ້ໃກ້ຄຽງທີ່ສຸດ, ຜູ້ແປຈຶ່ງນຳໃຊ້ສະບັບແປຫຼາຍສະບັບ ດັ່ງທີ່ຂຽນໄວ້ຂ້າງລຸ່ມ, ເຊິ່ງມີການກຳກັບຊື່ ຫຼື ລະຫັດພຣະຄຳພີ ເພື່ອຜູ້ອ່ານຈະຮູ້ທີ່ມາຂອງຂໍ້ຄວາມທີ່ອ້າງອີງ. ບາງຄັ້ງ, ຖ້າຂໍ້ຄວາມນັ້ນມາຈາກສະບັບລາວບູຮານ ຫຼື ສະບັບພາສາໄທ ຫຼື ພາສາອັງກິດ, ກໍຈະມີຄຳວ່າ “ດັດແປງຈາກສະບັບ...”, ເຊິ່ງການດັດແປງນັ້ນ ບໍ່ແມ່ນການປ່ຽນຄວາມໝາຍ ແຕ່ເປັນການປັບປຸງພາສາໃຫ້ເຂົ້າກັບພາສາລາວປັດຈຸບັນເທົ່າທີ່ເຮັດໄດ້.ສະບັບແປພຣະຄຳພີທີ່ໃຊ້ໃນເຫຼ້ມນີ້ ຄື:

    * **ພາສາລາວ**
        * ສະບັບຕີພິມ ຄ.ສ. 2015; (ບໍ່ມີການກຳກັບ)
        * ສະບັບປັບປຸງໃໝ່ ຄ.ສ. 2021; (LCV), [ສະບັບນີ້ມີການປັບປຸງອອນລາຍກ່ອນຕີພັມເລັກນ້ອຍ ແຕ່ຍັງຄົງໃຊ້ຊື່ເດີມ.]
        * ພຣະຄຳພີໃໝ່ລຸ້ນເກົ່າ ຄ.ສ. 1972; (LO1972)
        * ສະບັບລາວບູຮານ; (ສະບັບລາວບູຮານ)
    * **ພາສາໄທ**
        * ໄທຄິງເຈມ (TKJV)
        * ສະບັບຕີພິມ ຄ.ສ. 1972; (TH1971)
        * ອະມະຕະທຳລ່ວມສະໄໝ; (TNCV)
        * New Thai Version (ສະບັບແປໃໝ່); (NTV)
        * ສະບັບຕີພິມ ຄ.ສ. 2011; (THSV)
        * ສະບັບຕີພິມ ຄ.ສ. 1940; (TH1940)
        * ສະບັບອ່ານເຂົ້າໃຈງ່າຍ; (THA-ER)
    * **ພາສາອັງກິດ**
        * ສະບັບຄິງເຈມ (KJV)

**ການນຳໃຊ້ພາສາ**

ພາສາເປັນເລື່ອງທີ່ໃກ້ຕົວ ແລະ ສຳນວນຂອງແຕ່ລະຄົນກໍອີງໃສ່ປະສົບການຊີວິດຂອງຜູ້ນັ້ນ. ຜູ້ແປພະຍາຍາມເລືອກຄຳທີ່ໜ້າອ່ານເຂົ້າໃຈງ່າຍ ຕາມຂໍ້ຈຳກົດຄວາມສາມາດຂອງຕົນ ໂດຍຄຳນຶງເຖິງເປົ້າໝາຍຂອງການແປ ຄືການໃຊ້ສຳນວນທີ່ຜູ້ອ່ານສາມາດເຂົ້າເຖິງຄວາມໝາຍ ແລະ ເຈດຕະນາລົມຂອງຜູ້ຂຽນໄດ້. ໃນເລື່ອງການສະກົດຊື່ພາສາຕ່າງປະເທດນັ້ນ ມີຄວາມຄິດເຫັນຕ່າງໆນາໆ; (ຍົກຕົວຢ່າງຄຳວ່າ “Adventist” ມີວິທີສະກົດເປັນພາສາລາວເຖິງ 96 ວິທີ!) ຜູ້ແປຈຶ່ງນຳຄວາມຄິດເຫັນຈາກຫຼາຍຄົນປະກອບເຂົ້າໃນການກຳນົດຕົວສະກົດ ໂດຍສ່ວນຫຼາຍໃຫ້ຊື່ພາສາຝຣັ່ງອອກສຽງຕາມພາສາຝຣັ່ງ ແລະ ຊື່ພາສາອັງກິດ ແລະ ພາສາອື່ນໃຫ້ອອກສຽງຕາມພາສາອິງກິດ. ສ່ວນການໃຊ້ຈຸດຈ້ຳນັ້ນ ກໍເປັນອີກເລື່ອງໜຶ່ງທີ່ມີຄວາມຄິດເຫັນທີ່ແຕກຕ່າງກັນຫຼາຍ. ຜູ້ແປຈຶ່ງພະຍາຍາມໃຊ້ຈຸດຈ້ຳຕາມລະບົບສາກົນໃຫ້ຖືກຕ້ອງເທົ່າທີ່ເຮັດໄດ້ ໂດຍຮູ້ວ່າ ມີຫຼາຍບ່ອນທີ່ຍັງບໍ່ໄດ້ມາດຕະຖານ. 
ພາສາເໝືອນສິ່ງມີຊິວິດທີ່ມີການປ່ຽນແປງ ແລະ ພັດທະນາຢູ່ຕະຫຼອດເວລາ; ດັ່ງນັ້ນ, ຜູ້ຈັດພິມຈຶ່ງໃຊ້ຕົວສະກົດຕາມສະໄໝໃໝ່ເປັ່ນສ່ວນໃຫຍ່ຕາມທີ່ປາກົດຢູ່ໃນວັດຈະນານຸກົມຕ່າງໆ ໂດຍມີຂໍ້ຍົກເວັ້ນຢູ່(ບໍ່ຫຼາຍ/ບໍ່ເທົ່າໃດຄົ): ເຊັ້ນມີການໃຊ້ຣາຊາສັບ ແລະ ຕົວ “ຣ ຫັນຫຼີ້ນ” ຢູ່ໃນບາງຄຳ ແລະ ມີການໃຊ້ຕົວ “ສ” ໃນພຣະນາມຂອງພຣະຄຣິສ, ຄຣິສຕຽນ ແລະ ຄຣິສຕະຈັກ ເພື່ອເປັນການຖວາຍກຽດ. 

ຫາກຜູ້ໃດມີຄຳແນະນຳ ຫຼື ຂໍ້ເສໜີ ກໍຂໍໃຫ້ສົ່ງມາຍັງ ******* ເພື່ອຈະໄດ້ນຳໄປປັບປຸງສະບັບອອນລາຍ ແລະ ດັດແກ້ຕົ້ນສະບັບສຳລັບການຕີພິມຄັ້ງຕໍ່ໄປ. ສ່ວນສະບັບອອນລາຍນັ້ນກໍສາມາດເຂົ້າອ່ານໄດ້ທີ່ *******.
